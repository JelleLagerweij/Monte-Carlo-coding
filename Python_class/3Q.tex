\section{Results}
\subsection{Isotherms of liquid and gaseous methane}
The ultimate goal of this assignment is to be able to compute the classical thermodynamic state of methane: the relationship between $T$, $\rho$ -or equivalently $v$- and $P$. The code computes the canonical ensemble, so $\rho$ and $T$ are set naturally, while the pressure is computed. To test the dependants of the pressure an isochore in both liquid and gaseous state is simulated. As the algorithm can take quite some time to run, only few points in the phase space are assessed. These isochores are presented in different graphs, \cref{fig:Liquid_result} and \cref{fig:Gas_result}, as liquid and gaseous methane scale quite differently with their temperature.

\begin{figure}[th!]
	\centering
	\small
	\def\svgwidth{0.95\columnwidth}
	\input{Figures/IsochoreLiquid.pdf_tex}
	\caption{The isochore of methane at \SI{358.4}{\kg\per\m^3} simulated using \num{362} and \num{1000} particles compared to measured data. The simulated values correspondent nicely with the measured data. Although the simulation using \num{1000} particles performs somewhat better than the \num{362} particles simulation, both seem to predict the development of pressure as function of the temperature quite accurately.}
	\label{fig:Liquid_result}
\end{figure}

\begin{figure}[th!]
	\centering
	\small
	\def\svgwidth{0.95\columnwidth}
	\input{Figures/IsochoreGas.pdf_tex}
	\caption{The isochore of methane at \SI{1.6}{\kg\per\m^3} simulated using \num{362} and \num{1000} particles compared to measured data. The simulated values correspondent nicely with the measured data. For methane in its gaseouse phase, both simulations result in comparable, small, deviations from the measured values. However, the \num{1000} particles simulation results in significantly lower estimated error. Compared to \cref{fig:Liquid_result}, the isochore in gaseous phase is clearly linear, this is in accordance with ideal gas behaviour.}
	\label{fig:Gas_result}
\end{figure}

\subsection{The radial distribution function}
To get further understanding of molecular simulations and the fundamental difference between the liquid and gaseous phase, the radial distribution function is computed as well. To investigate the behaviour of these distributions in detail, the simulations with \num{3500} particles from \cref{tab:ResultsTableL} and \cref{tab:ResultsTableG} are used. The cut-off distance of these simulations was $L/2$, resulting in long, tails. The resulting graphs, \cref{fig:Rad_dist}, shows smooth behaviour. No discontinuities can be recognised in the radial distribution function. This indicates the correct implementation of the LJ potential and its corrections.

\begin{figure}[th!]
	\centering
	\small
	\def\svgwidth{0.95\columnwidth}
	\input{Figures/Long_probability_Density.pdf_tex}
	\caption{The radial distribution function of liquid and gaseous methane. As the ensemble exists of \num{3500} particles and \num{10.5e6} Monte Carlo trials are executed, the local errors in the probability density function are small.} 
	\label{fig:Rad_dist}
\end{figure}
