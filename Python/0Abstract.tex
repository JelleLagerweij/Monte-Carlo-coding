\begin{abstract}
    This report is written for assignment 1 of the introduction to molecular simulation course at the mechanical engineering masters track at the TU Delft. The main topic of this report is a python program which simulate a Lennard Jones fluid using a canonical Monte Carlo algorithm. A simulation of liquid and gaseous methane is used to show the performance of this code. A simple TraPPE forcefield is used to model the interactions between the methane molecules. The intramolecular interactions are not taken into account in this force field, which only considers LJ interactions between the molecules. As the goal of this report is to show my personal development and the choices made while coding my program in python, I will keep the language informal.
    
    The resulting code is optimised for ensembles ranging from \numrange{250}{1000} particles with a large cut-off distance. This results in smooth radial distribution functions with long tails. Array operations in numpy are used to increase the computational efficiency as much as possible and the observables -potential energy, pressure and the radial distribution function- are sampled every so often to reduce their effect on the computation time. The example of modelling methane shows that the code is able to accurately predict the pressure in both the liquid and gaseous phase and is able to clearly show the radial distribution function that is expected of these phases. The produced code exists of two files: the class file ('LJ\_Monte\_Carlo.py'), which is the general canonical Monte Carlo program, and an example execute file ('execute.py') that shows the practical use of the class file for methane.
\end{abstract}