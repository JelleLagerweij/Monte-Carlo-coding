\section{Retrieving results}
\subsection{Assessing step size for liquid methane}
Setting the optimal maximum step size, to stay in a acceptance ratio of \SIrange{40}{50}{\percent}, by trial and error is prone to errors. Especially if these step sizes are determined when non-equilibrated configurations are used. To solve this problem, a function is build that updates the maximum step size according to the success rate of the last few trials. This function uses the acceptance rate, which is kept track of in the Monte Carlo function, \cref{Monte Carlo}. The function, as shown in \cref{max step size}, increases the maximum step size if the acceptance rate is too high, and reduces the maximum step size if the acceptance rate is too low.

\begin{listing}[ht!]
	\begin{minted}{python}
def newStepsize(self):
	acceptance = self.acceptance
	max_step = self.max_step

	if acceptance <= 0.1:
	max_step *= 0.2
	elif acceptance < 0.4 and acceptance > 0.1:
	max_step *= 0.8
	elif acceptance > 0.5 and acceptance < 0.9:
	max_step *= 1.2
	elif acceptance >= 0.9:
	max_step *= 2
	
	if max_step > self.Rcut/10:
	max_step = self.Rcut/10
	\end{minted}
\caption{The function which optimises the step size.}
	\label{max step size}
\end{listing}

To still answer this question in the assignment appropriately, the development of the maximum step size and the acceptance rate are tracked during the part of the algorithm that equilibrates the system. For 362 particles, it results in \cref{fig:EquilibratingLiquid}. From this, I conclude that the advised max step size is quite close to the optimal maximum step size. During the main part of the Monte Carlo algorithm, the step size keeps being updated, however it stays close to \SI{0.5}{\angstrom}.

\begin{figure}[th!]
	\centering
	\small
	\def\svgwidth{0.95\columnwidth}
	\input{Figures/EquilibratingLiquid.pdf_tex}
	\caption{The development of the acceptance rate and the maximum step size during the equilibration phase for the liquid phase.} 
	\label{fig:EquilibratingLiquid}
\end{figure}

\subsection{Assessing step size for gaseous methane}
For the gaseous state of methane, at \SI{400}{\kelvin} and \SI{9.68}{\kg\per\meter^3}, an interesting observation is made. The step size does not seem to influence the acceptance rate any more. This is because the density is so low, that the Lennard-Jones interactions are close to zero. A replaced particle will therefore still be likely to not interact with other particles by much. With the automatic maximum step size adjustment function, this behaviour can lead to exponential growth of the maximum step size. To keep some physical logic in the development of the state of the system, a maximum step size of $r_\text{cut}/10$ is used. The choice to make this limit dependant of $r_\text{cut}$ instead of $\sigma$ or $L$ was arbitrary. This is just implemented to keep the maximum step size under an upper limit.

\begin{figure}[th!]
	\centering
	\small
	\def\svgwidth{0.95\columnwidth}
	\input{Figures/EquilibratingGas.pdf_tex}
	\caption{The development of the acceptance rate and the maximum step size during the equilibration phase for the gaseous phase. It is clear that the acceptance rate lies far above the \SI{50}{\percent}, but the maximum step size is kept constant. This is because the density of the ensemble is so low that moving a particle to any place in the box results in a $\approx\SI{90}{\percent}$ acceptance rate. Therefore, no max step size that result in a \SIrange{40}{50}{\percent} acceptance rate exists.}
	\label{fig:EquilibratingGas}
\end{figure}

\subsection{Checking if the method equilibrates correctly}
To check if the code calibrates correctly, the model for \num{1000} particles is run twice, and the energy is plotted in \cref{fig:Eq zoomed out} and \cref{fig:Eq zoomed in}. The first time, the equilibration phase is turned of, while the second computations is equilibrated first before simulating. Without equilibrating, the average result for the energy and the pressure are "out of control", reaching values of $E_\text{tot}=\SI{3(3)e4}{\zepto\joule}$ and $P=\SI{2(1)e8}{\bar}$. If the set is equilibrated first, the values are more logical, resulting in $E_\text{tot}=\SI{-9.889(5)}{\zepto\joule}$ and $P=\SI{31(4)}{\bar}$. 

\begin{figure}[th!]
	\centering
	\small
	\def\svgwidth{0.95\columnwidth}
	\input{Figures/Equilibrating_zoomed_out.pdf_tex}
	\caption{Zoomed out, no details are seen in the fluctuations. However, it is clear that initially the simulation without equilibration first has much larger energies, although these go down quickly. The simulation with initial equilibration clearly stays constant on this scale of energies. The results without equilibration seems to be vertical only, but it has a horizontal part as well, the blue line ended up behind the orange one.}
	\label{fig:Eq zoomed out}
\end{figure}

\begin{figure}[th!]
	\centering
	\small
	\def\svgwidth{0.95\columnwidth}
	\input{Figures/Equilibrating_zoomed_in.pdf_tex}
	\caption{When the scale of the energy is zoomed in to discern the energy fluctuations, it is clear that except the initial trial moves, both models converge to a similar energy level they fluctuate around. However, the initial peak of the non-equilibrated influences the averages too much to be useful.}
	\label{fig:Eq zoomed in}
\end{figure}


\subsection{Comparing different ensemble sizes}
To compare the effects of the amount of particles on the pressure, the energy and the computation time, the model is ran for $N=$ 362, 1000 and 3500 for three thermodynamically different cases. Case one, shown in \cref{tab:ResultsTableL}, was performed on a liquid state, \SI{150}{\K} at \SI{358.4}{\kg\per\meter^3}. The other cases was in gaseous phase at \SI{400}{\K} at \SI{9.68}{\kg\per\meter^3} and its results are shown in \cref{tab:ResultsTableG}. The observables were sampled \num{1500} times, for each simulation, and sampled every $2N$ trial moves.

\begin{table}[ht!]
	\centering
	\begin{tabular}{|c|ccc|}
		\hline
		$N$ & \multicolumn{1}{c}{$E$ per particle in \si{\zepto\joule}} & \multicolumn{1}{c}{$P$ in \si{\bar}} & C-time in \si{\s} \\ \hline
		3500& \num{-9.933(2)}   &  \num{21(2)}&  \num{10300}  	\\
		1000& \num{-9.890(4)}	&  \num{34(4)}&   \num{1020}  	\\ 
		362 & \num{-9.681(5)}	&  \num{39(5)}&   \num{160}		\\ \hline
	\end{tabular}
	\caption{The Liquid phase results. Each measurements consists of \num{1500} observables sampled every $2N$ trial moves. The cut-off distance is set to $L/2$ for the larger ensembles, for the ensemble with 362 particles a cut-off distance of \SI{14}{\angstrom} used. For these results, the shift and tail corrections are used. The computation time is recorded as well (C-time).}
	\label{tab:ResultsTableL}
\end{table}

\begin{table}[ht!]
	\centering
	\begin{tabular}{|c|ccc|}
		\hline
		$N$ &  \multicolumn{1}{c}{$E$ per particle in \si{\zepto\joule}} & \multicolumn{1}{c}{$P$ in \si{\bar}} &  C-time in \si{\s} \\ \hline
		3500& \num{-0.2392(3)}  &  \num{19.896(1)} 	& \num{6800} \\
		1000& \num{-0.2379(4)}	&  \num{19.9016(6)} & \num{1090}  \\
		362 & \num{-0.2387(7)}	& \num{19.90(1)}	& \num{220}	\\ \hline
	\end{tabular}
	\caption{The Gaseous phase results. The cut-off distance is set to $L/2$ for the larger ensembles, for the ensemble with 362 particles a cut-off distance of \SI{30}{\angstrom} used. For these results, the shift and tail corrections are used. The computation time is recorded as well (C-time).}
	\label{tab:ResultsTableG}
\end{table}

From these results it can be recognised that the pressure is more dependant on the number of particles and the error of this observable is larger than that in the potential energy of the ensemble. Besides that, the computational efficiency seems to be quite good, only taking \SI{200}{\s} of computation time for the case with \num{362} particles.